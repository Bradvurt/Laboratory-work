\documentclass[13pt, a4paper]{scrartcl}
\usepackage[russian]{babel}
\usepackage[a4paper,top=2cm,bottom=2cm,left=3cm,right=3cm,marginparwidth=1.75cm]{geometry}
\usepackage{amsmath}
\pagestyle{empty}


\begin{document}
\parindent0pt
ет, что эвольвента кривой описывается точкой прямой, катящейся без скольжения по этой кривой.
\setlength{\parindent}{20pt}

Свойство $2^0$ эволюты и эвольвенты дает возможность вычислять длину дуг эволюту, если известны радиусы кривизн эвольвенты. Найдем этим методом длину одной арки циклоиды (см. примеры в п. 16.2, 16.4 и 16.5).

В примере 3 п. 17.5 было показано, что для радиуса кривизны $R$ циклоиды $x=r(t-\sin t), \, y=r(1-\cos t), \, 0 \leq t \leq 2\pi$, справедлива формула 

\[R=R(t)=4r \sin \frac{t}{2} \]
и что эволютой циклоиды является та же самая циклоида, но несколько сдвинутая. Поэтому длина половины арки циклоиды, соответствующей изменению параметра от $0$ до $\pi$ (на ней радиус кривизны возрастает), равен $R(\pi)-R(0)=4r$. Следовательно, длина всей арки циклоиды равна $8r$.

\subsection*{17.7. Кручение пространственной кривой}

Плоские кривые полностью с точностью до положения в пространстве описываются своей кривизной. Именно в дифференциальной геометрии доказыввается, что для всякой непрерывной неотрицательной функции $k(s), \, 0 \leq s \leq S$, можно построить единственную с точностью до ее положения в пространстве плоскую кривую, для которой заданная функция является кривизной (см.: \emph{Рашевский П. К.} Курс дифференциальной геометрии. - М.: ГИТТЛ, 1956).

Пространственные же кривые полностью описываются с помощью кривизны и так называемого кручения. Для его определения введем понятие бинормали.

Рассмотрим пространственную кривую $\Gamma= \{r(s); \, 0 \leq s \leq S\} $, где $s$ - переменная длина дуги.
\\ \textbf{Определение 9.} \textit{Векторное произведение единичного касательного вектора \textbf{t} и главной нормали \textbf{n} в данной точке уривой называется бинормалью кривой в этой точке.}

Бинормаль обозначается через \textit{\textbf{b}}. Таким образом,

\begin{equation*}
    \textit{\textbf{b}} \overset{def}{=} \textit{\textbf{t}} \times \textit{\textbf{n}}. \eqno( 17.41 )
\end{equation*}

Очевидно, что бинормаль определена в тех точках, в которых определена главная нормаль, т.е. в которых кривизна не равна нулю.

Тройка единичных взаимно перпендикулярных векторов \textit{\textbf{t}}, \textit{\textbf{n}} и \textit{\textbf{b}} называется основным репером или, менее точно, основным трехгранником кривой в данной точке.
\begin{center}
    \line(1, 0){100} \\
    447
\end{center}

\end{document}
